\documentclass[]{article}
\usepackage{hyperref}

\setcounter{section}{-1}

\usepackage{titlesec}

\setcounter{secnumdepth}{4}

\titleformat{\paragraph}
{\normalfont\normalsize\bfseries}{\theparagraph}{1em}{}
\titlespacing*{\paragraph}
{0pt}{3.25ex plus 1ex minus .2ex}{1.5ex plus .2ex}

\begin{document}

\title{The Shire Constitution}
\author{}
\date{}
\maketitle
\tableofcontents
\newpage

\section{Preamble}
We, the Hobbits of The Shire, do establish this Constitution on the Twenty-fifth of February, MMXV, to govern the matters of our establishment and facilitate harmony amongst its Hobbits.

\section{Article I - Purpose}
\subsection{}
The purpose of this establishment, or apartment, is to ensure the common welfare and protect the natural Nig (used as a verb) rights of its citizens, also known as housemates, also known as Hobbits.
\subsection{}
This Constitution, ratified, serves as the supreme Law of the Shire.
\subsubsection{}
All previously enacted verbal agreements, unless amended through Constitutional means, remain in full force and effect.

\section{Article II - Executive}
\subsection{Presidency}
\subsubsection{}
The President is the sole bearer of executive power.
\subsubsection{}
The President may be referred to as Chief Nig.  
\subsubsection{}
The President’s laws, conforming to the Constitution, override any Legislative action.
\subsection{Defined Powers}
\subsubsection{}
The President may enact Executive Orders at his or her own discretion, without deliberation by the Legislature.
\subsubsection{}
Executive Orders carry the same weight and authority as legislation passed by the Legislature. 
\subsubsection{}
Executive Orders may be repealed by a unanimous vote by the Legislature.
\subsubsection{}
The President holds Veto power over any bill passed by the Legislature. 
\subsection{Responsibilities}
\subsubsection{}
The President is responsible for organizing and delegating house duties, and also for enforcing full participation of all Hobbits in these tasks.
\subsubsection{}
House duties include but are not limited to the following: chores, timely payment of bills, cooking etiquette, etc.
\subsubsection{}
The President is responsible for arbitrating disputes between Hobbits. The President has the option to defer this responsibility to the Legislature.
\subsection{Term Length}
\subsubsection{}
One Presidential term lasts two (2) weeks.
\subsubsection{}
The outgoing President’s executive power ceases once the incoming President has been elected.
\subsubsection{}
The authority of the incoming President’s executive power takes effect once he, she, or ze has been elected.
\subsubsection{}
The President may resign from office at any point during his or her term. 

\section{Article III - Legislative}
\subsection{Membership}
\subsubsection{}
The Legislature consists of all Hobbits except Chief Nig.
\subsection{Defined Powers}
\subsubsection{}
The Legislature may pass legislation, equal in weight and authority to any Presidential law with three (3) votes.
\subsubsection{}
Legislation passed by the Legislature can only be repealed by three (3) votes in the Legislature
\subsubsection{}
The Legislature may Veto legislation enacted by the President with three (3) votes.
\subsection{Impeachment}
\subsubsection{}
The Legislature may call a Vote of Impeachment against the Executive at any time.
\subsubsection{}
The Executive shall be expelled from his or her office with a unanimous vote from the Legislature only.
\subsubsection{}
If the Presidency is vacant, the Legislature shall appoint a Hobbit to serve the remainder of the existing term, assuming all rights, powers, and responsibilities of the Executive during his or her time in office.
\section{Article IV - Elections}
\subsection{Frequency}
\subsubsection{}
Elections shall be held every two weeks, specifically at the end of every executive term.
\subsection{Manner}
\subsubsection{}
Elections will be conducted via secret ballot. Vote count shall be subject to oversight by a legislatively appointed Auditor.
\subsubsection{}
Voting is compulsory. Abstaining is not acceptable. The President is responsible for ensuring full participation.
\subsubsection{}
The President shall be elected through the Approval Voting system outlined in the video \href{https://www.youtube.com/watch?v=orybDrUj4vA}{``Quick and Easy Voting for Normal People" by CGPGrey}.
\paragraph{}
The procedures for deliberation immediately prior to the Election shall be decided by general consensus.

\section{Article V - Ratification}
\subsection{}
Ratification of this Constitution requires unanimous vote.

\section{Article VI - Amendments}
\subsection{}
Amendments are additions and/or changes to the articles of this Constitution.
\subsubsection{}
In the event of possible legal conflict between a) an Amendment and a Constitutional Article or b) two amendments, the most recent ratified amendment is recognized as law.
\subsection{}
Amendments are to be ratified by unanimous vote.
\subsubsection{}
All other means of altering the content and/or meaning of this Constitution are unlawful and subject to investigation by the Auditor of The Shire.

\end{document}